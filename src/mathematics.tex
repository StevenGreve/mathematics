% !TEX spellcheck=en_US
\documentclass[a4paper]{article}
\usepackage[american]{babel}
\usepackage[T1]{fontenc}
\usepackage[utf8]{inputenc}
\usepackage{graphicx}
\usepackage{abstract}
\usepackage{hyperref}
\usepackage{mathtools}
\usepackage{amssymb}
\usepackage{tikz}
\usepackage{witharrows}
\usetikzlibrary{calc, arrows.meta, bending, decorations.pathreplacing}

\renewcommand{\abstractnamefont}{\normalfont\large\bfseries}
\newcommand{\bref}[1]{(\ref{#1})}

\newtheorem{definition}{Definition}

\begin{document}

\begin{titlepage}
    \centering
    {\scshape\Huge \bfseries Mathematics \par}
    \par\vspace{1.5cm}
    \includegraphics[width=0.25\textwidth]{img/placeholder.png}\par
    \vspace{1.5cm}
    {\huge\bfseries Study Notes \par}
    \vspace{1cm}
    {\LARGE\itshape Steven Bryan Greve \par}
    \vspace{1cm}
    {\large\today\par}
    \vfill
    \begin{abstract}
        This document pertains to mathematical concepts which I have studied. It includes my notes, work, and solutions when dealing with various problems. The source code for this document can be found on my \href{https://github.com/StevenGreve}{\emph{GitHub}} at any time.
    \end{abstract}
\end{titlepage}

\newpage

\tableofcontents

\newpage

\section{Definitions}

% BINOMIAL THEOREMS 

\vspace{0.5cm}\subsection{Binomial Theorems}

\vspace{0.5cm}\begin{definition}\label{def-biomial-theorems}

    \begin{equation}
        (a+b)^2 = a^2 + 2ab + b^2 \label{eq-1}
    \end{equation}

    \begin{equation}
        (a-b)^2 = a^2 - 2ab + b^2 \label{eq-2}
    \end{equation}

    \begin{equation}
        (a+b)\,(a-b) = a^2 - b^2 \label{eq-3}
    \end{equation}

    \flushleft \normalfont For higher exponentiations:  

    \begin{equation}
        (a+b)^3 = a^3 +3a^2b + 3ab^2 + b^3 \label{eq-4}
    \end{equation}

    \begin{equation}
        (a-b)^3 = a^3 - 3a^2b + 3ab^2 - b^3 \label{eq-5}
    \end{equation}

    \begin{equation}
        (-a-b)^3 = -a^3 - 3a^2b - 3ab^2 - b^3 \label{eq-6}
    \end{equation}

    Binomial Formula:

    \begin{equation}
        (x+y)^{n}=\sum _{k=0}^{n}{n \choose k}x^{n-k}y^{k}=\sum _{k=0}^{n}{n \choose k}x^{k}y^{n-k} \label{eq-7}
    \end{equation}

    where

    \begin{equation}
        {\displaystyle {\binom {n}{k}}={\frac {n!}{k!(n-k)!}},} \label{eq-8}
    \end{equation}
    
\end{definition}

% FRACTIONS 

\vspace{0.5cm}\subsection{Fractions}

\vspace{0.5cm}\begin{definition}\label{def-fractions}

    \begin{equation}
        \frac{a}{b} + \frac{c}{b} = \frac{a + c}{b} \label{eq-9}
    \end{equation}

    \begin{equation}
        \frac{a}{b} - \frac{c}{b} = \frac{a - c}{b} \label{eq-10}
    \end{equation}

    \begin{equation}
        \frac{a}{b} \cdot  \frac{c}{d} = \frac{a \cdot c}{b \cdot d} \label{eq-11}
    \end{equation}

    \begin{equation}
        \frac{a}{b} + \frac{c}{d} = \frac{ad}{bd} + \frac{bc}{bd} = \frac{ad + bc}{bd} \label{eq-12}
    \end{equation}

    \begin{equation}
        \frac{a}{b} - \frac{c}{d} = \frac{ad}{bd} - \frac{bc}{bd} = \frac{ad - bc}{bd} \label{eq-13}
    \end{equation}

    \flushleft \normalfont Inverse: 

    \begin{equation}
        \frac{a}{b} \div \frac{c}{d} = \frac{a}{b} \cdot \frac{d}{c} = \frac{ad}{bc} \label{eq-14}
    \end{equation}

    \begin{equation}
        \dfrac{\dfrac{a}{b}}{\dfrac{c}{d}} = \frac{a}{b} \cdot \frac{d}{c} = \frac{a \cdot b}{d \cdot c} \label{eq-15}
    \end{equation}

    \begin{equation}
        \dfrac{a \cdot \dfrac{b}{c}}{\dfrac{d}{e}} = \dfrac{\dfrac{a \cdot c + b}{c}}{\dfrac{d}{c}} = \dfrac{a \cdot c + b}{c} \cdot \dfrac{c}{d} = \dfrac{(a \cdot c + b) \cdot c}{c \cdot d} \label{eq-16}
    \end{equation}

    \begin{equation}
        \frac{ab}{c} = \frac{a}{c} \cdot b \label{eq-17}
    \end{equation}

    \begin{equation}
        \frac{a}{b} = \frac{1}{b} \cdot a \label{eq-18}
    \end{equation}

    \begin{equation}
        \frac{a \div b}{c} = \frac{a}{c} \div b \label{eq-19}
    \end{equation}
    
\end{definition}

% PARENTHESES

\vspace{0.5cm}\subsection{Parentheses Rules}

\vspace{0.5cm}\begin{definition}\label{def-parentheses}
    
    \begin{equation}
    +\,(a+b) = a+b \label{eq-20}
    \end{equation}

    \begin{equation}
        +\,(-a-b) = -a-b \label{eq-21}
    \end{equation}

    \begin{equation}
        -\,(a-b) = -a+b \label{eq-22}
    \end{equation}

    \begin{equation}
        -\,(-a+b) = +a-b \label{eq-23}
    \end{equation}

    \begin{equation}
        -\,(a+b) = -a-b \label{eq-24}
    \end{equation}

    \flushleft \normalfont Associative properties:

    \begin{equation}
        (a+b)+c = a+(b+c) \label{eq-25}
    \end{equation}

    \begin{equation}
        (a \cdot b) \cdot c = a \cdot (b \cdot c) \label{eq-26}
    \end{equation}

    Distributive properties: 

    \begin{equation}
        a \cdot (b+c) = (a \cdot b) + (a \cdot c) \label{eq-27}
    \end{equation}

    \begin{equation}
        (a+b) \cdot c = (a \cdot c) + (b \cdot c) \label{eq-28}
    \end{equation}

    Commutative properties:

    \begin{equation}
        a+b = b+a \label{eq-29}
    \end{equation}

    \begin{equation}
        a \cdot b = b \cdot a \label{eq-30}
    \end{equation}

\end{definition}

% MULTIPLY WITH -1

\vspace{0.5cm}\subsection{Multiply with -1}
\vspace{0.5cm}\begin{definition}\label{def-multiply-with-minus-one}
    
    Mathematical operators may be swapped by multiplying with -1, because the result does not change.
    
    \begin{equation}
         a+b = c\; \;\boldsymbol{\leftrightarrow}\; \;-1 \cdot (-a-b) = c \label{eq-31}
    \end{equation}

    \flushleft \normalfont \bfseries Example: 

    \begin{equation}
        (a-b)^2 = (b-a)^2 \label{eq-32}
    \end{equation}

\end{definition}

% SQUARE ROOTS 

\vspace{0.5cm}\subsection{Square Roots}
\vspace{0.5cm}\begin{definition}\label{def-square-roots}

    \begin{equation}
        \sqrt[1]{a} = a \label{eq-33}
    \end{equation}

    \begin{equation}
        \sqrt[2]{a} = \sqrt{a} \label{eq-34}
    \end{equation}

    \begin{equation}
        \sqrt{a^2} = a \label{eq-35}
    \end{equation}

    \begin{equation}
        \left(\sqrt{a}\right)^2 = a \label{eq-36}
    \end{equation}
    
    \begin{equation}
        \frac{1}{\sqrt{n}} \cdot \frac{1}{\sqrt{n}} = \frac{1}{n} \label{eq-37}
    \end{equation}

    \begin{equation}
        \sqrt{n} \cdot \sqrt{n} = n \label{eq-38}
    \end{equation}

    \flushleft \normalfont Addition: 

    \begin{equation}
        a\sqrt[n]{x} + b\sqrt[n]{x} = (a+b)\sqrt[n]{x} \label{eq-39}
    \end{equation}

    Subtraction: 

    \begin{equation}
        a\sqrt[n]{x} - b\sqrt[n]{x} = (a-b)\sqrt[n]{x} \label{eq-40}
    \end{equation}

    Multiplication: 

    \begin{equation}
        \sqrt[n]{a} \cdot \sqrt[n]{b} = \sqrt[n]{a \cdot b} \label{eq-41}
    \end{equation}

    Division: 

    \begin{equation}
        \dfrac{\sqrt[n]{a}}{\sqrt[n]{b}} = \sqrt[n]{\dfrac{a}{b}} \label{eq-42}
    \end{equation}

    Root exponentiation:

    \begin{equation}
       \Big(\sqrt[n]{a}\Big)^m = \sqrt[n]{m} \label{eq-43}
    \end{equation}

    Root extraction: 

    \begin{equation}
        \sqrt[m]{\sqrt[n]{a}} = \sqrt[m \cdot n]{a} \label{eq-44}
    \end{equation}

    Transforming roots into exponents:

    \begin{equation}
        \sqrt[n]{a} = a \cdot \frac{1}{n} \label{eq-45}
    \end{equation}

    \begin{equation}
        \sqrt{a} = a \cdot \frac{1}{2} \label{eq-46}
    \end{equation}

    \begin{equation}
        \sqrt[n]{a^m} = a^\frac{m}{n} \label{eq-47}
    \end{equation}

\end{definition}

% EXPONENTIATION

\vspace{0.5cm}\subsection{Exponentiation}
\vspace{0.5cm}\begin{definition}\label{def-exponentiation}
    
    \begin{equation}
        x^n \cdot x^b = x^{n+b} \label{eq-48}
    \end{equation}

    \begin{equation}
        x^n \div x^b = \frac{x^n}{x^b} = x^{n-b} \label{eq-49}
    \end{equation}

    \begin{equation}
       \Big(x^a\Big)^b = x^{a \cdot b} \label{eq-50}
    \end{equation}

    \begin{equation}
        a^n \cdot a^b = (a \cdot b)^n \label{eq-51}
    \end{equation}

    \begin{equation}
        a^n \div b^n = \frac{a^n}{b^n} = \left(\frac{a}{b}\right)^n \label{eq-52}
    \end{equation}

    \begin{equation}
        x^0 = 1 \label{eq-53}
    \end{equation}

    \begin{equation}
        x^1 = x \label{eq-54}
    \end{equation}

    \begin{equation}
        x^{-n} = \frac{1}{x^n} \label{eq-55}
    \end{equation}

    \begin{equation}
        \frac{1}{x} = x^{-1} \label{eq-56}
    \end{equation}

    \begin{equation}
        x^{\frac{1}{n}} = \sqrt[n]{x} \label{eq-57}
    \end{equation}

    \flushleft \normalfont {\bfseries Disclaimer (for 56):} If n is even, then x must be > 0!

    \begin{equation}
        x^{\frac{m}{n}} = \sqrt[n]{m} \label{eq-58}
    \end{equation}

    \begin{equation}
        x^{-\frac{m}{n}} = \frac{1}{\sqrt[n]{x^m}} \label{eq-59}
    \end{equation}

    Addition: %(cf. \ref{eq-power-to-sqrt})

    \begin{equation}
        ax^n + bx^n = (a+b)x^n \label{eq-60}
    \end{equation}

    Subtraction: 

    \begin{equation}
        ax^n - bx^n = (a-b)x^n \label{eq-61}
    \end{equation}

    Transform a single root into a exponent:

    \begin{equation}
        \sqrt{a} = (a)^{\frac{1}{2}} \cdots \sqrt[3]{a} = (a)^{\frac{1}{3}} \cdots \label{eq-62}
    \end{equation}

\end{definition}      

\section{Calculus}

\vspace{0.5cm}\subsection{Simplifying Algebraic Terms}

 % \Arrow[xoffset=-1cm]{Text}

 % Exercise 1

 \vspace{0.5cm}Exercise 1: 

 \begin{equation}
     \setlength{\jot}{10pt}
     \begin{WithArrows}
        [12x+5x \cdot 2-(10x-8x)]+18x \div 3 & \Arrow{Apply parentheses rule \bref{eq-22}} \\
        =[12x+5x \cdot 2-10x+8x]+18x \div 3 & \Arrow{Substract $-10x+8$} \\
        = [12x+5x \cdot 2-2x]+18x \div 3 & \Arrow{Multiply $5x \cdot 2$} \\
        = 20x+18x \div 3 \\
        = 20x+6x \\
        = 26x
     \end{WithArrows}
     \nonumber
 \end{equation}

 \flushleft Exercise 2:

 \begin{equation}
     \setlength{\jot}{10pt}
     \begin{WithArrows}
         x-((x-4)-(14+2x))+1 & \Arrow{Apply parentheses rule \bref{eq-24}} \\
         \Leftrightarrow x-(x-4-14-2x)+1 & \Arrow{\bref{eq-22}} \\
         \Leftrightarrow x-x+4+14+2x+1 \\ 
         \Leftrightarrow 19+2x & \Arrow{\bref{eq-29}} \\
         \Leftrightarrow 2x+19
     \end{WithArrows}
     \nonumber
 \end{equation}

 \newpage 

\vspace{0.5cm}\subsection{Square Root Equations}

% Exercise 1

\vspace{0.5cm} Exercise 1: 

\begin{equation}
    \setlength{\jot}{10pt}
    \begin{WithArrows}
        2x + \sqrt{x^2 + 9} = 4x +3 & \Arrow{$-2x$} \\
        \sqrt{x^2 + 9} = 2x+3 & \Arrow{$(\;)^2$} \\
        x^2 + 9 = (2x+3)^2 & \Arrow{Apply first binomial rule \bref{eq-1}} \\
        x^2 + 9 = 4x^2 + 2x \cdot 2 \cdot 3 + 9 & \Arrow{Summarize} \\ 
        x^2 + 9 = 4x^2 + 12x + 9 & \Arrow{$-x^2-9$} \\
        0 = 3x^2 + 12x \\
        0 = \underbrace{x}_\text{$x_1$}\underbrace{(3x+12)}_\text{$x_2$}
    \end{WithArrows}
    \nonumber
\end{equation}

\begin{equation}
    \Rightarrow x_1 = 0
    \quad \mid \quad \text{Verify for $x_1$:} \quad 2 \cdot 0 + \sqrt{0^2 + 9} = 3
    \nonumber
\end{equation}

\begin{equation}
    \Rightarrow x_2 = 3x+12 = 0 \quad \Leftrightarrow \quad x_2 = -4
    \quad \mid \quad \text{Verify for $x_2$:} \quad 4 \cdot (-4) + 3 = -13
    \nonumber
\end{equation}

\begin{equation}
    \Rightarrow 3 = -13
    \nonumber
\end{equation}

\begin{equation}
    Q.E.D
    \nonumber
\end{equation}

% Exercise 2

Exercise 2: 

\begin{equation}
    \setlength{\jot}{5pt}
    \begin{WithArrows}
        \sqrt{x+2} + \sqrt{x-1} = 3 & \Arrow{$(\;)^2$} \\
        (\sqrt{x+2} + \sqrt{x-1})^2 = 3^2 & \Arrow{Apply first binomial rule \bref{eq-1}} \\
        \sqrt{x+2}^2 + 2\cdot\sqrt{x+2}\cdot\sqrt{x-1} + \sqrt{x-1}^2 = 9 & \Arrow{Square, and apply rule \bref{eq-35}} \\
        x+2 + 2\sqrt{(x+2)\cdot(x-1)} + x-1 = 9 & \Arrow{$x+2 + x-1$} \\ 
        2x+1 + 2\sqrt{(x+2)\cdot(x-1)} = 9 & \Arrow{$-2x-1$} \\
        2\sqrt{(x+2)\cdot(x-1)} = -2x+8 & \Arrow{$\div 2$} \\
        \sqrt{(x+2)\cdot(x-1)} = -x+4 & \Arrow{Apply third binomial rule \bref{eq-3}} \\
        \sqrt{x^2-1x+2x-2} = -x+4 & \Arrow{$(\;)^2$} \\
        x^2-1x+2x-2 = (-x+4)^2 & \Arrow{No binomial rule applies for $(-x+4)^2$} \\
        x^2+1x-2 = x^2-2\cdot4x+16 & \Arrow{$-x^2-1x+2$} \\
        0 = -9x+18 & \Arrow{$+9x$} \\
        9x = 18 & \Arrow{$\div9$} \\
        x = 2
    \end{WithArrows}
    \nonumber
\end{equation}

% Exercise 3

Exercise 3:

\begin{align*}
    \sqrt{-3x-1-\sqrt{4x+5}}&=1 &&\text{$(\;)^2$}&\\[1.25ex]
    -3x-1-\sqrt{4x+5}&=1 &&\text{$+3x+1$}&\\[1.25ex]
    -\sqrt{4x+5}&=3x+2 &&\text{$(\;)^2$}&\\[1.25ex]
    4x+5 &= (3x+2)^2 &&\text{Apply first binomial rule \bref{eq-1}}&\\[1.25ex]
    4x+5 &= 9x^2 + 2 \cdot 3x \cdot 2 + 4&\\[1.25ex]
    4x+5 &= 9x^2 + 12x + 4 &&\text{$-4x-5$}&\\[1.25ex]
    0 &= 9x^2 + 8x - 1 &&\text{$\div 9$}&\\[1.25ex]
    0 &= x^2 + \frac{8}{9}\cdot x - \frac{1}{9} &&\text{Apply \textbf{PQ} formula}&\\[1.25ex]
    x_{1,2}&=-{\frac {p}{2}}\pm {\sqrt {\left({\frac {p}{2}}\right)^{2}-q}}&&\\[1.25ex]
    x_1 &\Leftrightarrow \sqrt{-\frac{11}{3}} = 1 &\\[1.25ex]
    x_2 &\Leftrightarrow 1 = 1 &\\[2ex]
    \Rightarrow L &= \{-1\} &
\end{align*}

% Exercise 4

Exercise 4:

\begin{align*}
    x = \sqrt{x+20} &&\text{$x^2$}&\\[1.25ex]
    x^2 = x+20 &&\text{$-x-20$}&\\[1.25ex]
    \Leftrightarrow x^2-x-20 = 0&\\[1.25ex]
    \Leftrightarrow (x-5)(x+4) = 0&\\[1.25ex]
    \Rightarrow x = 5 \quad \land x = -4&\\[1.25ex]
    \Rightarrow \sqrt{5+20} = 5 \quad \mid \quad \sqrt{-4+20} = \sqrt{16} = 4&\\[1.25ex]
    \Rightarrow L = \{5\} &
\end{align*}

% Changing the Subject of an Equation 

\newpage 

\vspace{0.5cm}\subsection{Changing the Subject of an Equation }

% Exercise 1

\vspace{0.5cm} Exercise 1: 

\begin{align*}
    \frac{6x+7}{9} - \frac{10x+7}{18} = \frac{9x+5}{14} - \frac{9x-16}{20} &&\text{Find lowest common denominator.}&\\[1.25ex]
    \Leftrightarrow \frac{6x+7}{2 \cdot 9} - \frac{10x+7}{18} =\frac{9x+5}{14 \cdot 10} - \frac{9x-16}{20 \cdot 7} &\\[1.25ex]
    \Leftrightarrow  \frac{12+14}{18} - \frac{10x+7}{18} = \frac{90x+50}{140} - \frac{63x-112}{140} &\\[1.25ex]
    \Leftrightarrow  \frac{12x+14-(10x+7)}{18} = \frac{90x+50-(63x-112)}{140} &\\[1.25ex]
    \Leftrightarrow  \frac{12x+14-10x-7}{18} = \frac{90x+50-63x+112}{140} &\\[1.25ex]
    \Leftrightarrow \frac{2x+7}{18} = \frac{27x+162}{140} &&\text{$\cdot 18$}&\\[1.25ex]
    \Leftrightarrow  2x+7 = \frac{(27x+162)\cdot18}{140} &&\text{Reduce fraction with $2$.}&\\[1.25ex]
    \Leftrightarrow 2x+7 = \frac{(27x+162)9}{70} &&\text{$\cdot 70$}&\\[1.25ex]
    \Leftrightarrow 140x+490 = 9(27x+162)&\\[1.25ex]
    \Leftrightarrow 140x+490 = 243x+1458 &&\text{$-140x$}&\\[1.25ex]
    \Leftrightarrow 490 = 103x + 1458 &&\text{$-1458$}&\\[1.25ex]   
    \Leftrightarrow -968 = 103x &&\text{Apply \bref{eq-29}}&\\[1.25ex]
    \Leftrightarrow 103x = -968 &&\text{$\div 103$}&\\[1.25ex]
    \Leftrightarrow x = -\frac{968}{103}&
\end{align*}

% Exercise 2

\newpage Exercise 2: 

\begin{align*}
    \frac{3x-9}{6x-1} = \frac{4x-16}{8x-5} &&\text{$\cdot(8x-5) \cdot(6x-1)$}&\\[1.25ex]
    \Leftrightarrow (3x-9)\cdot(8x-5)=(4x-16)\cdot(6x-1)&\\[1.25ex]
    \Leftrightarrow 24x^2-15x-72x+45 = 24x^2-4x-96x+16 &&\text{$-24x^2$}&\\[1.25ex]
    \Leftrightarrow -15x-72x+45=-4x-96x+16 &&\text{$-45$}&\\[1.25ex]
    \Leftrightarrow -15x-72x = -4x-96x-29&\\[1.25ex]
    \Leftrightarrow -87x = -100x-29&&\text{$+100x$}&\\[1.25ex]
    \Leftrightarrow 13x = -29&&\text{$\div13$}&\\[1.25ex]
    \Leftrightarrow x = -\frac{29}{13}&
\end{align*}

% Exercise 3

\begin{align*}
    x^2+(x-2)^2 = 10 &\\[1.25ex]
    \Leftrightarrow x^2 + (x-2)(x-2)=10&\\[1.25ex]
    \Leftrightarrow x^2+x^2-2x-2x+4=10&\\[1.25ex]
    \Leftrightarrow 2x^2-4x+4 = 10 &&\text{$-10$}&\\[1.25ex]
    \Leftrightarrow 2x^2-4x-6 = 0 &&\text{$\div 2$}&\\[1.25ex]
    \Leftrightarrow x^2-2x-3 = 0 &&\text{Apply \textbf{PQ} formula.}&\\[1.25ex]
    \Leftrightarrow x_1 = 3 \quad \mid \quad x_2 = -1 &\\[1.25ex]
    \Rightarrow L = \{3\} &
\end{align*}


\newpage 

% Exponential Equations

\vspace{0.5cm}\subsection{Exponential Equations}

\vspace{0.5cm}Exercise 1:

\begin{align*}
    \frac{(2^{-4})^{-5}\cdot2^{17}}{(2^{-3})^{-6}\cdot(2^{-4})^3} &&\text{Apply \bref{eq-50}.} &\\[1.25ex]
    = \frac{2^{(-4)\cdot(-5)}\cdot 2^{17}} {2^{(-3)\cdot(-6)}\cdot2^{(-4)\cdot3 }}&\\[1.25ex]
    = \frac{2^{20} \cdot 2^{17}}{2^{18} \cdot 2^{-12}} &&\text{Apply \bref{eq-48} and \bref{eq-49}}&\\[1.25ex]
    = 2^{20 + 17 - 18 -(-12)} &\\[1.25ex]
    = 20^{20 + 17 -18 + 12} &\\[1.25ex]
    = 20^{31} &
\end{align*}

Exercise 2:

\begin{align*}
    \frac{3^7 \cdot (3^{-2})^3}{3^{-4} \cdot 3^7} \div \frac{(3^4)^{-3}}{(3^{-2})^{-6}} &&\text{Apply rule \bref{eq-50}} &\\[1.25ex]
    = \frac{3^7 \cdot 3^{(-2) \cdot 3}}{3^{-4} \cdot 3^7} \div \frac{3^{4 \cdot {(-3)}}}{3^{(-2) \cdot (-6)}} &&\text{Apply rule \bref{eq-14} and \bref{eq-48}} &\\[1.25ex]
    = \frac{3^7 \cdot 3^{-6}}{3^{-4+7}} \cdot \frac{3^{12}}{3^{-12}} &\\[1.25ex]
    = \frac{3^7 \cdot 3^{-6} \cdot 3^{12}}{3^3 \cdot 3^{-12}}&&\text{Apply \bref{eq-48} and \bref{eq-49}}&\\[1.25ex]
    = 3^{7-6+12-3-(-12)}&\\[1.25ex]
    = 3^{22}&
\end{align*}

Exercise 3:

\begin{align*}
    \frac{12x^{-2} y^3}{8z^2} \cdot \frac{4y^{-2}z}{3x^{-5}} \div \frac{6z^{-3}}{2y^{-4}z} &&\text{Apply rule \bref{eq-14}.}&\\[1.25ex]
    =\frac{12x^{-2} y^3}{8z^2} \cdot \frac{4y^{-2}z}{3x^{-5}} \cdot \frac{2y^{-4}z}{6z^{-3}}&&\text{Reduce $\frac{12\cdot4\cdot2}{8\cdot3\cdot6}$ and apply \bref{eq-48} and \bref{eq-49}.}&\\[1.25ex]
    =\frac{2}{3}x^{-2-(-5)}\cdot y^{3-2-4}\cdot z^{1+1-2-(-3)}&\\[1.25ex]
    =\frac{2}{3}x^3 y^{-3} z^3 &&\text{Apply rule \bref{eq-55}.}&\\[1.25ex]
    =\frac{2}{3}\frac{x^3 z^3}{y^3}&
\end{align*}

Exercise 4:

\begin{align*}
    \frac{2^4 \cdot x^5 \cdot y^7 \cdot z^8}{8 \cdot x^2 \cdot y^5 \cdot z^{10}} \div \frac{2 \cdot x^2 \cdot y^5 \cdot z^8}{16 \cdot x^4 \cdot y^3 \cdot z^5}&&\text{Apply rule \bref{eq-14}.}&\\[1.25ex]
    = \frac{2^4 \cdot x^5 \cdot y^7 \cdot z^8}{8 \cdot x^2 \cdot y^5 \cdot z^{10}} \cdot \frac{16 \cdot x^4 \cdot y^3 \cdot z^5}{2 \cdot x^2 \cdot y^5 \cdot z^8}&&\text{Reduce $\frac{2}{8} \cdot \frac{16}{2}$ and apply \bref{eq-48}.}&\\[1.25ex]
    = \frac{2}{1}\cdot \frac{x^{5+4}\cdot y^{7+3}\cdot z^{8+5}}{x^{2+2}\cdot y^{5+5}\cdot z^{10+8}}&\\[1.25ex]
    = \frac{2}{1}\cdot \frac{x^9\cdot y^{10}\cdot z^{13}}{x^4 \cdot y^{10}\cdot z^{18}} &&\text{Apply rule \bref{eq-49}.}&\\[1.25ex]
    = 2\cdot x^5 \cdot z^{-5} &&\text{Apply rule \bref{eq-55}.}&\\[1.25ex]
    = \frac{2\cdot x^5}{z^5}&
\end{align*}

Exercise 5:

\begin{align*}
    \frac{4x^{2-m}y^{3\cdot m}}{7z^{m-n}} \div \frac{5z^{m+n}x^{3-m}}{14y^{1-2m}}&&\text{Apply rule \bref{eq-14}.}&\\[1.25ex]
    = \frac{4x^{2-m}y^{3\cdot m}}{7z^{m-n}} \cdot \frac{14y^{1-2m}}{5z^{m+n}x^{3-m}}&&\text{Reduce $\frac{4}{7} \cdot \frac{14}{5}$ and apply \bref{eq-48} and \bref{eq-49}.}&\\[1.25ex]
    = \frac{8}{5}\cdot \frac{x^{2-m-(3-m)}\cdot y^{3m+1-2m}}{z^{m-n+m+n}}&\\[1.25ex]
    = \frac{8}{5}\cdot \frac{x^{-1}\cdot y^{m+1}}{z^{2m}}&&\text{Apply rule \bref{eq-55}.}&\\[1.25ex]
    = \frac{8}{5} \frac{y^{m+1}}{x\cdot z^{2m}}&
\end{align*}

\newpage

% Square Root Equations 

\vspace{0.5cm}\subsection{Square Root Equations}

\vspace{0.5cm}Exercise 1:

\begin{align*}
    a^{\frac{2}{5}} \cdot \sqrt[5]{a^3} &&\text{Apply rule \bref{eq-46}.}&\\[1.25ex]
    a^{\frac{2}{5}} \cdot (a^3)^{\frac{1}{5}} &&\text{Apply rule \bref{eq-50}.}&\\[1.25ex]
    = a^{\frac{2}{5}} \cdot a^{\frac{3}{5}} &&\text{Apply rule \bref{eq-48}.}&\\[1.25ex]
    = a^{\frac{2}{5} + \frac{3}{5}}&\\[1.25ex]
    = a^1 &&\text{Apply rule \bref{eq-54}.}&\\[1.25ex]
    = a &
\end{align*}

Exercise 2:

\begin{align*}
    &\sqrt{a \sqrt[3]{a^2}} \quad \div \quad \left(a\sqrt{a^{-3}\sqrt{a^{-1}}}\right)&&\text{Apply rule \bref{eq-47}.}&\\[1.25ex]
    &= \sqrt{a^1 \cdot a^{\frac{2}{3}}} \quad \div \quad \left(a^1 \cdot \sqrt{a^{-3} \cdot a^{-\frac{1}{2}}}\right) &&\text{Apply rule \bref{eq-48} and \bref{eq-62}.}&\\[1.25ex]
    &= \left(a^{1+\frac{2}{3}}\right)^\frac{1}{2} \quad \div \quad \left(a^1 \cdot \left(a^{-3-\frac{1}{2}}\right)^\frac{1}{2}\right)&&\text{Apply rule \bref{eq-50}.}&\\[1.25ex]
    &= a^{\frac{5}{3}\cdot \frac{1}{2}} \quad \div \quad \left(a^1 \cdot a^{-\frac{7}{2}\cdot \frac{1}{2}}\right)&&\text{Simplify and apply rule \bref{eq-48}.}&\\[1.25ex]
    &= a^{\frac{5}{6}} \div a^{1-\frac{7}{4}}&\\[1.25ex]
    &= a^{\frac{5}{6}} \div a^{-\frac{3}{4}}&&\text{Apply rule \bref{eq-49}.}&\\[1.25ex]
    &= a^{\frac{5}{6}-(-\frac{3}{4})}&\\[1.25ex]
    &= a^{\frac{19}{12}}
\end{align*}

Exercise 3: 
\begin{align*}
    \sqrt[3]{(a^2)^{-5} \cdot \sqrt[4]{a^{16}}} &&\text{Apply rule \bref{eq-62}, \bref{eq-50}, and \bref{eq-47}.}&\\[1.25ex]
    = \left(a^{-10} \cdot a^{\frac{16}{4}}\right)^{\frac{1}{3}}&\\[1.25ex]
    = \left(a^{-10} \cdot a^4\right)^{\frac{1}{3}}&&\text{Apply rule \bref{eq-48}.}&\\[1.25ex]
    = \left(a^{-10+4}\right)^{\frac{1}{3}}&\\[1.25ex]
    = \left(a^{-6}\right)^{\frac{1}{3}}&&\text{Apply rule \bref{eq-50} and \bref{eq-55}.}&\\[1.25ex]
    = a^{-2}
    = \frac{1}{a^2}
\end{align*}

Exercise 4:
\begin{align*}
    &\frac{1}{\sqrt[3]{a\sqrt[5]{a^{-20}}}} &&\text{Apply rule \bref{eq-62} and \bref{eq-47}.}&\\[1.25ex]
    &= \frac{1}{\left(a^1 \cdot a^{\frac{-20}{5}}\right)^{\frac{1}{3}}}&&\text{$\frac{-20}{5}$ equals $-4$.}&\\[1.25ex]
    &= \frac{1}{\left(a^1 \cdot a^{-4}\right)^{\frac{1}{3}}}&&\text{Apply rule \bref{eq-48}.}&\\[1.25ex]
    &= \frac{1}{\left(a^{1-4}\right)^{\frac{1}{3}}}
    = \frac{1}{(a^{-3})^{\frac{1}{3}}} &&\text{Apply rule \bref{eq-50}.}&\\[1.25ex]
    &= \frac{1}{a^{-1}} = a^1 = a
\end{align*}

Exercise 5:
\begin{align*}
    &\sqrt[3]{a\sqrt{a}} \div \sqrt{a^{-3}\sqrt[4]{a^6}} &&\text{Apply rule \bref{eq-62} and \bref{eq-47}.}&\\[1.25ex]
    &= \left(a^1 \cdot a^{\frac{1}{2}}\right)^{\frac{1}{3}} \div \left(a^{-3} \cdot a^{\frac{6}{4}}\right)^{\frac{1}{2}}&&\text{eq-50}&\\[1.25ex]
    &= \left(a^{\frac{1}{3}} \cdot a^{\frac{1}{6}}\right) \div \left(a^{-\frac{3}{2}} \cdot a^{\frac{3}{4}}\right)&&\text{Apply rule \bref{eq-48}.}&\\[1.25ex]
    &= \left(a^{\frac{1}{3}+\frac{1}{6}}\right) \div \left(a^{\frac{-3}{2}+\frac{3}{4}}\right) = a^{\frac{1}{2}} \div a^{\frac{-3}{4}}&&\text{Apply rule \bref{eq-49}.}&\\[1.25ex]
    &= a^{\frac{1}{2}-\left(\frac{-3}{4}\right)} = a^{\frac{1}{2} + \frac{3}{4}}&\\[1.25ex]
    &= a^{\frac{5}{4}}
\end{align*}

\include{sections/inequalities/inequalities}
\section{Ring Theory}

In the near future...

\end{document}