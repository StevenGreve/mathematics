\section{Definitions}

% BINOMIAL THEOREMS 

\vspace{0.5cm}\subsection{Binomial Theorems}

\vspace{0.5cm}\begin{definition}\label{def-biomial-theorems}

    \begin{equation}
        (a+b)^2 = a^2 + 2ab + b^2 \label{eq-1}
    \end{equation}

    \begin{equation}
        (a-b)^2 = a^2 - 2ab + b^2 \label{eq-2}
    \end{equation}

    \begin{equation}
        (a+b)\,(a-b) = a^2 - b^2 \label{eq-3}
    \end{equation}

    \flushleft \normalfont For higher exponentiations:  

    \begin{equation}
        (a+b)^3 = a^3 +3a^2b + 3ab^2 + b^3 \label{eq-4}
    \end{equation}

    \begin{equation}
        (a-b)^3 = a^3 - 3a^2b + 3ab^2 - b^3 \label{eq-5}
    \end{equation}

    \begin{equation}
        (-a-b)^3 = -a^3 - 3a^2b - 3ab^2 - b^3 \label{eq-6}
    \end{equation}

    Binomial Formula:

    \begin{equation}
        (x+y)^{n}=\sum _{k=0}^{n}{n \choose k}x^{n-k}y^{k}=\sum _{k=0}^{n}{n \choose k}x^{k}y^{n-k} \label{eq-7}
    \end{equation}

    where

    \begin{equation}
        {\displaystyle {\binom {n}{k}}={\frac {n!}{k!(n-k)!}},} \label{eq-8}
    \end{equation}
    
\end{definition}

% FRACTIONS 

\vspace{0.5cm}\subsection{Fractions}

\vspace{0.5cm}\begin{definition}\label{def-fractions}

    \begin{equation}
        \frac{a}{b} + \frac{c}{b} = \frac{a + c}{b} \label{eq-9}
    \end{equation}

    \begin{equation}
        \frac{a}{b} - \frac{c}{b} = \frac{a - c}{b} \label{eq-10}
    \end{equation}

    \begin{equation}
        \frac{a}{b} \cdot  \frac{c}{d} = \frac{a \cdot c}{b \cdot d} \label{eq-11}
    \end{equation}

    \begin{equation}
        \frac{a}{b} + \frac{c}{d} = \frac{ad}{bd} + \frac{bc}{bd} = \frac{ad + bc}{bd} \label{eq-12}
    \end{equation}

    \begin{equation}
        \frac{a}{b} - \frac{c}{d} = \frac{ad}{bd} - \frac{bc}{bd} = \frac{ad - bc}{bd} \label{eq-13}
    \end{equation}

    \flushleft \normalfont Inverse: 

    \begin{equation}
        \frac{a}{b} \div \frac{c}{d} = \frac{a}{b} \cdot \frac{d}{c} = \frac{ad}{bc} \label{eq-14}
    \end{equation}

    \begin{equation}
        \dfrac{\dfrac{a}{b}}{\dfrac{c}{d}} = \frac{a}{b} \cdot \frac{d}{c} = \frac{a \cdot b}{d \cdot c} \label{eq-15}
    \end{equation}

    \begin{equation}
        \dfrac{a \cdot \dfrac{b}{c}}{\dfrac{d}{e}} = \dfrac{\dfrac{a \cdot c + b}{c}}{\dfrac{d}{c}} = \dfrac{a \cdot c + b}{c} \cdot \dfrac{c}{d} = \dfrac{(a \cdot c + b) \cdot c}{c \cdot d} \label{eq-16}
    \end{equation}

    \begin{equation}
        \frac{ab}{c} = \frac{a}{c} \cdot b \label{eq-17}
    \end{equation}

    \begin{equation}
        \frac{a}{b} = \frac{1}{b} \cdot a \label{eq-18}
    \end{equation}

    \begin{equation}
        \frac{a \div b}{c} = \frac{a}{c} \div b \label{eq-19}
    \end{equation}
    
\end{definition}

% PARENTHESES

\vspace{0.5cm}\subsection{Parentheses Rules}

\vspace{0.5cm}\begin{definition}\label{def-parentheses}
    
    \begin{equation}
    +\,(a+b) = a+b \label{eq-20}
    \end{equation}

    \begin{equation}
        +\,(-a-b) = -a-b \label{eq-21}
    \end{equation}

    \begin{equation}
        -\,(a-b) = -a+b \label{eq-22}
    \end{equation}

    \begin{equation}
        -\,(-a+b) = +a-b \label{eq-23}
    \end{equation}

    \begin{equation}
        -\,(a+b) = -a-b \label{eq-24}
    \end{equation}

    \flushleft \normalfont Associative properties:

    \begin{equation}
        (a+b)+c = a+(b+c) \label{eq-25}
    \end{equation}

    \begin{equation}
        (a \cdot b) \cdot c = a \cdot (b \cdot c) \label{eq-26}
    \end{equation}

    Distributive properties: 

    \begin{equation}
        a \cdot (b+c) = (a \cdot b) + (a \cdot c) \label{eq-27}
    \end{equation}

    \begin{equation}
        (a+b) \cdot c = (a \cdot c) + (b \cdot c) \label{eq-28}
    \end{equation}

    Commutative properties:

    \begin{equation}
        a+b = b+a \label{eq-29}
    \end{equation}

    \begin{equation}
        a \cdot b = b \cdot a \label{eq-30}
    \end{equation}

\end{definition}

% MULTIPLY WITH -1

\vspace{0.5cm}\subsection{Multiply with -1}
\vspace{0.5cm}\begin{definition}\label{def-multiply-with-minus-one}
    
    Mathematical operators may be swapped by multiplying with -1, because the result does not change.
    
    \begin{equation}
         a+b = c\; \;\boldsymbol{\leftrightarrow}\; \;-1 \cdot (-a-b) = c \label{eq-31}
    \end{equation}

    \flushleft \normalfont \bfseries Example: 

    \begin{equation}
        (a-b)^2 = (b-a)^2 \label{eq-32}
    \end{equation}

\end{definition}

% SQUARE ROOTS 

\vspace{0.5cm}\subsection{Square Roots}
\vspace{0.5cm}\begin{definition}\label{def-square-roots}

    \begin{equation}
        \sqrt[1]{a} = a \label{eq-33}
    \end{equation}

    \begin{equation}
        \sqrt[2]{a} = \sqrt{a} \label{eq-34}
    \end{equation}

    \begin{equation}
        \sqrt{a^2} = a \label{eq-35}
    \end{equation}

    \begin{equation}
        \left(\sqrt{a}\right)^2 = a \label{eq-36}
    \end{equation}
    
    \begin{equation}
        \frac{1}{\sqrt{n}} \cdot \frac{1}{\sqrt{n}} = \frac{1}{n} \label{eq-37}
    \end{equation}

    \begin{equation}
        \sqrt{n} \cdot \sqrt{n} = n \label{eq-38}
    \end{equation}

    \flushleft \normalfont Addition: 

    \begin{equation}
        a\sqrt[n]{x} + b\sqrt[n]{x} = (a+b)\sqrt[n]{x} \label{eq-39}
    \end{equation}

    Subtraction: 

    \begin{equation}
        a\sqrt[n]{x} - b\sqrt[n]{x} = (a-b)\sqrt[n]{x} \label{eq-40}
    \end{equation}

    Multiplication: 

    \begin{equation}
        \sqrt[n]{a} \cdot \sqrt[n]{b} = \sqrt[n]{a \cdot b} \label{eq-41}
    \end{equation}

    Division: 

    \begin{equation}
        \dfrac{\sqrt[n]{a}}{\sqrt[n]{b}} = \sqrt[n]{\dfrac{a}{b}} \label{eq-42}
    \end{equation}

    Root exponentiation:

    \begin{equation}
       \Big(\sqrt[n]{a}\Big)^m = \sqrt[n]{m} \label{eq-43}
    \end{equation}

    Root extraction: 

    \begin{equation}
        \sqrt[m]{\sqrt[n]{a}} = \sqrt[m \cdot n]{a} \label{eq-44}
    \end{equation}

    Transforming roots into exponents:

    \begin{equation}
        \sqrt[n]{a} = a \cdot \frac{1}{n} \label{eq-45}
    \end{equation}

    \begin{equation}
        \sqrt{a} = a \cdot \frac{1}{2} \label{eq-46}
    \end{equation}

    \begin{equation}
        \sqrt[n]{a^m} = a^\frac{m}{n} \label{eq-47}
    \end{equation}

\end{definition}

% EXPONENTIATION

\vspace{0.5cm}\subsection{Exponentiation}
\vspace{0.5cm}\begin{definition}\label{def-exponentiation}
    
    \begin{equation}
        x^n \cdot x^b = x^{n+b} \label{eq-48}
    \end{equation}

    \begin{equation}
        x^n \div x^b = \frac{x^n}{x^b} = x^{n-b} \label{eq-49}
    \end{equation}

    \begin{equation}
       \Big(x^a\Big)^b = x^{a \cdot b} \label{eq-50}
    \end{equation}

    \begin{equation}
        a^n \cdot a^b = (a \cdot b)^n \label{eq-51}
    \end{equation}

    \begin{equation}
        a^n \div b^n = \frac{a^n}{b^n} = \left(\frac{a}{b}\right)^n \label{eq-52}
    \end{equation}

    \begin{equation}
        x^0 = 1 \label{eq-53}
    \end{equation}

    \begin{equation}
        x^1 = x \label{eq-54}
    \end{equation}

    \begin{equation}
        x^{-n} = \frac{1}{x^n} \label{eq-55}
    \end{equation}

    \begin{equation}
        \frac{1}{x} = x^{-1} \label{eq-56}
    \end{equation}

    \begin{equation}
        x^{\frac{1}{n}} = \sqrt[n]{x} \label{eq-57}
    \end{equation}

    \flushleft \normalfont {\bfseries Disclaimer (for 56):} If n is even, then x must be > 0!

    \begin{equation}
        x^{\frac{m}{n}} = \sqrt[n]{m} \label{eq-58}
    \end{equation}

    \begin{equation}
        x^{-\frac{m}{n}} = \frac{1}{\sqrt[n]{x^m}} \label{eq-59}
    \end{equation}

    Addition: %(cf. \ref{eq-power-to-sqrt})

    \begin{equation}
        ax^n + bx^n = (a+b)x^n \label{eq-60}
    \end{equation}

    Subtraction: 

    \begin{equation}
        ax^n - bx^n = (a-b)x^n \label{eq-61}
    \end{equation}

    Transform a single root into a exponent:

    \begin{equation}
        \sqrt{a} = (a)^{\frac{1}{2}} \cdots \sqrt[3]{a} = (a)^{\frac{1}{3}} \cdots \label{eq-62}
    \end{equation}

\end{definition}      
